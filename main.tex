\documentclass[10pt,a4paper, leqno]{scrartcl}
\usepackage[utf8]{inputenc}
\usepackage{amsmath}
\usepackage{amsfonts}
\usepackage{amssymb}
\usepackage{fancyhdr}
\usepackage{lastpage}
\usepackage{tikz}
\usepackage{algpseudocode}
\usetikzlibrary{arrows}
\usepackage{algorithm}
\usepackage{caption}
\usepackage{xcolor}
\usepackage[autostyle]{csquotes}
\usepackage{graphicx} 
\usepackage{listings}
\usepackage{color}
\usepackage{geometry}
\usepackage{eurosym}
\usepackage{tabu}
\usepackage{booktabs}
\usepackage{lmodern}
\usepackage{tabu}
\usepackage{graphicx}
\usepackage{wrapfig}
\usepackage[german]{babel}
\usepackage{titlesec}
\usepackage{multirow}
\usepackage{venndiagram}
\usepackage[european]{circuitikz}


\newcommand{\lheader}{Linker Header }
\newcommand{\rheader}{Rechter Header }
\newcommand{\titletext}{Titel}


\titleformat*{\section}{\large\bfseries}
\titleformat*{\subsection}{\normalsize\bfseries}

\geometry{a4paper,left=12mm,right=12mm, top=15mm, bottom=20mm} 

\newcommand{\vars}{\texttt}
\newcommand{\func}{\textrm}
\let\oldReturn\Return
\renewcommand{\Return}{\State\oldReturn}


\algnewcommand\algorithmicforeach{\textbf{for each}}
\algdef{S}[FOR]{ForEach}[1]{\algorithmicforeach\ #1\ \algorithmicdo}

\pagestyle{fancy}
\fancyhf{}

\lhead{\lheader}
\rhead{\rheader}
\renewcommand{\footrulewidth}{0.4pt} %untere Trennlinie
\rfoot{\pagemark}



% RS Shortcuts
\newcommand{\xor}{\oplus}
\renewcommand{\and}{\land}
\renewcommand{\or}{\lor}
\renewcommand{\not}[1]{\overline{#1}}
\newcommand{\eq}{\Leftrightarrow}
\newcommand{\dnot}[1]{\not{\not{#1}}}
\newcommand{\dec}{_{10}}
\newcommand{\bi}{_2}

% Section Number
\setcounter{section}{1}


\begin{document}

\begin{center}
  \textbf{\huge{\titletext}} \\
\end{center}

% OBDD
\begin{tikzpicture}
  % Ebene 0
  \node[circle,draw] (n0) at (2,0) {$x_1$};
  % Ebene 1
  \node[circle,draw] (n1_1) at (1,-1.5) {$x_2$};
  \node[circle,draw] (n1_2) at (3,-1.5) {$x_3$};
  % Pfeile 0->1
  \path[->,dashed,line width=0.2mm] (n0) edge [left] node {0} (n1_1);
  \path[->,line width=0.2mm] (n0) edge [right] node {1} (n1_2);
  % Ebene 2
  \node[circle,draw] (n2_1) at (1,-3) {$x_3$};
  \node[circle,draw] (n2_2) at (3,-3) {$x_2$};
  % Pfeile 1->2
  \path[->,dashed,line width=0.2mm] (n1_1) edge [left] node {0} (n2_1);
  \path[->,line width=0.2mm] (n1_2) edge [right] node {1} (n2_2);
  \path[->,line width=0.2mm] (n1_1) edge [left] node {1} (n2_2);
  % Senken
  \node[rectangle,draw] (0) at (1,-4.5) {$0$};
  \node[rectangle,draw] (1) at (3,-4.5) {$1$};
  % Pfeile 2->S
  \path[->,line width=0.2mm](n2_1) edge [left] node {1} (0);
  \path[->,line width=0.2mm] (n2_2) edge [right] node {1} (1);
  \path[->,dashed,line width=0.2mm] (n2_2) edge  (0);
  \draw(1.7,-3.5)node[anchor=south]{\small{$0$}};
  \path[->,dashed,line width=0.2mm] (n2_1) edge  (1);
  \draw(2.3,-3.5)node[anchor=south]{\small{$0$}};
\end{tikzpicture}

% Venn Diagram
\begin{venndiagram2sets}[showframe=false]
  \fillOnlyB
\end{venndiagram2sets}

% Schaltnetz
\begin{circuitikz}[
    input/.style    = {anchor=base,draw,circle,inner sep=1pt},
    dot/.style    = {anchor=base,fill,circle,inner sep=1pt},
    notdot/.style    = {anchor=base,fill,circle,inner sep=2pt},
    scale=0.7, every node/.style={transform shape}
  ]

  % Eingänge für 2 Variablen
  % Erzeugt Variable $x_1$ und $\overline{x_1}$
  \draw
  (-1, 13) node[not port] (not1) {}
  (-1, 11) node[not port] (not0) {}
  ;
  \draw
  (-4,14) node[input] {} --
  (-3.5,14) node[dot] {} node[above] {$x_1$} --
  (-3.5, 13) -- (not1.in) (not1.out) -- (0.0, 13) node[dot] {}

  (-3.5,14) -- (0, 14) node[dot] {}


  (-4,12) node[input] {} --
  (-3.5,12) node[dot] {} node[above] {$x_2$} --
  (-3.5, 11) -- (not0.in) (not0.out) -- (0, 11) node[dot] {}

  (-3.5,12) -- (0, 12) node[dot] {}
  ;

  % Fixes an der Norm von CircuitTikz, damit es näher am Skript ist
  % für die Punkte am Ausgang der not Gatter
  \draw
  (-1.1, 11.05) node[notdot] {}
  (-1.1, 13.05) node[notdot] {}
  ;

  % Und-Gatter  
  \draw
  (7, 12) node[and port] (and1) {}

  (7, 10) node[and port] (and2) {}
  ;

  % Oder Gatter
  \draw
  (10.5, 11.0) node[or port] (or1) {}
  ;


  % Verbindungen zwischen den Gattern
  \draw
  (and1.out) -- (or1.in 1)
  (and2.out) -- (or1.in 2)
  ;


  % Ein JK FlipFlop
  \draw
  (15,8) node[shape=jkff] (jk1) {$JK_1$}
  ;


  % OR-Ausgang zu FlipFlop Eingang 
  \draw
  (or1.out) -- (or1.out |- jk1.K) -- (jk1.K)
  ;

\end{circuitikz}



\end{document}

